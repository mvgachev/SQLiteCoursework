\documentclass{article}
\usepackage[utf8]{inputenc}
\usepackage{float}

\title{Coursework-Data Managment}
\author{mg5u19 }
\date{May 2020}

\begin{document}

\maketitle

\section{The Relational Model}

\subsection{EX1}
\begin{table}[H]
    \centering
    \begin{tabular}{|c|c|}
        \hline
         Data Set & Data Type  \\
         \hline
         dateRep & int  \\
         \hline
         day & int  \\
         \hline
         month & int    \\
         \hline
         year & int     \\
         \hline
         cases & int    \\
         \hline
         deaths & int   \\
         \hline
         countriesAndTerritories & string   \\
         \hline
         geoId & string     \\
         \hline
         countryterritoryCode & string      \\
         \hline
         popData2018 & int      \\
         \hline
         continentExp & string      \\
         \hline
    \end{tabular}
    \caption{Dataset Table}
    \label{tab:my_label}
\end{table}

\subsection{EX2}
Functional Dependencies:
\newline
\newline
$\{dateRep\}  \longleftarrow \longrightarrow  \{day,month,year\}$
\newline
\newline
$\{countriesAndTerritoryCode,geoId\}  \longrightarrow  \{countriesAndTerritories,continentExp,popData2018\}$
\newline
\newline
$\{countriesAndTerritoryCode,dateRep\}  \longrightarrow  \{cases,deaths,\}$
\newline
\newline

\subsection{EX3}
Potential candidate keys:
\newline
geoId and countryterritoryCode.
\newline
\newline

\subsection{EX4}
Primary key:
\newline
The primary key should be \{countriesAndTerritories, dataRep\} because it is compact, unique and definite for every country. From it we can access the country code, continent, popData, geoId, cases and deaths.
\newline
\newline

\section{Normalisation}
\subsection{EX5}
Partial key dependencies:
\newline
\newline
\{countriesAndTerritories\} $\longrightarrow$ \{continentExp,geoID,popData2018,countryTerritoryCode\}
\newline
\newline
$\{dateRep\}  \longrightarrow  \{day,month,year\}$
\newline
\newline
Additional Relations:
\newline
\newline
\{countriesAndTerritories,dateRep\} $\longrightarrow$ \{cases,deaths\}
\newline
\newline
\subsection{EX6}
In order to convert the relation into  a 2nd normal form we need to distribute the data into three tables: one table with key countriesAndTerritories and attributes countryTerritoryCode, continentExp, geoId and popData2018, another with key dateRep and attributes day, month and year, and last one with key {dateRep, countriesAndTerritories} with attributes cases and deaths.
\newline
\newline
\subsection{EX7}
Transitional Dependencies:
\newline
\newline
\{geoId\} \longrightarrow \{continentExp,countryTerritoryCode, popData2018\}
\newline
\newline
\{countryTerritoryCode\} \longrightarrow \{popData2018\}
\newline
\newline
\subsection{EX8}

At the end I have 5 tables: 
\newline
\begin{enumerate}
    \item Date Table - \{dateRep, day, month, year\}, Primary Key - dateRep
    \item Cases Table - \{dateRep, countriesAndTerritories, cases, deaths\}, Primary Key - dateRep and countriesAndTerritories
    \item Countries Table - \{countriesAndTerritories, geoId\}, Primary Key - countriesAndTerritories
    \item Locations Table - \{geoId, countryTerritoryCode, ContinentExp\}, Primary Key - geoId
    \item Population Table - \{countryTerritoryCode, popData2018\}, Primary Key - countryTerritoryCode
\end{enumerate}

    I distributed the data in that way because it is the most optimal way to access and find the information needed. I decided to separate the countryTerritoryCode and popData2018 in a new table because when I don't have a code, there is no popData2018 available.
    \newline
    \newline
\subsection{EX9}
Yes, my relation is in Boyce-Codd Normal Form because I don't have any partial key dependencies, nor any transitive dependencies. Also, the non-prime keys are only dependent on the prime ones.
\newline
\newline

\section{Modelling}
\newline
\subsection{EX11}
I have included an index on the countriesAndTerritories column on the Cases table in order to search for the country I need faster and more efficient.
\newline
\section{Querying}
\newline
\subsection{EX14}
SELECT SUM(cases),SUM(deaths)
FROM Cases
\newline
\subsection{EX15}
SELECT dateRep,cases\newline
FROM Cases\newline
WHERE countriesAndTerritories = 'United\_Kingdom'\newline
ORDER BY SUBSTR(dateRep,7,4),SUBSTR(dateRep,4,2),\newline
         SUBSTR(dateRep,1,2) ASC;\newline
\newline
\subsection{EX16}
SELECT Locations.continentExp,Cases.dateRep,\newline
    SUM(Cases.cases) AS sumOfCases,SUM(Cases.deaths) AS sumOfDeaths\newline
FROM Cases\newline
LEFT JOIN Countries\newline
    ON Cases.countriesAndTerritories = Countries.countriesAndTerritories\newline
LEFT JOIN Locations\newline
    ON Locations.geoId = Countries.geoId\newline
GROUP BY continentExp,dateRep\newline
ORDER BY SUBSTR(dateRep,7,4),SUBSTR(dateRep,4,2),\newline
         SUBSTR(dateRep,1,2) ASC;
\newline
\subsection{EX17}
SELECT Cases.countriesAndTerritories,\newline
       CAST(SUM(Cases.cases)AS FLOAT)/CAST(Populations.popData2018 AS FLOAT)*100\newline
       AS percentageOfCases,\newline
       CAST(SUM(Cases.deaths)AS FLOAT)/CAST(Populations.popData2018 AS FLOAT)*100\newline
       AS percentageOfDeaths\newline
FROM Cases\newline
LEFT JOIN Countries\newline
    ON Cases.countriesAndTerritories = Countries.countriesAndTerritories\newline
LEFT JOIN Locations\newline
    ON Countries.geoId = Locations.geoId\newline
LEFT JOIN Populations\newline
    ON Populations.countryTerritoryCode = Locations.countryTerritoryCode\newline
GROUP BY Cases.countriesAndTerritories;\newline
\subsection{EX18}
SELECT countriesAndTerritories,\newline
       CAST(deaths AS FLOAT)/CAST(cases AS FLOAT) AS percentDeathsOfPopulation\newline
FROM Cases\newline
ORDER BY percentDeathsOfPopulation DESC\newline
LIMIT 10;\newline
\subsection{EX19}
SELECT dateRep,countriesAndTerritories,cases,deaths,\newline
       SUM (cases) OVER(ORDER BY SUBSTR(dateRep,7,4),\newline
            SUBSTR(dateRep,4,2),\newline
            SUBSTR(dateRep,1,2) ASC) AS sumOfCases,\newline
       SUM(deaths) OVER(ORDER BY SUBSTR(dateRep,7,4),\newline
            SUBSTR(dateRep,4,2),\newline
            SUBSTR(dateRep,1,2) ASC) AS sumOfDeaths\newline
FROM Cases\newline
WHERE countriesAndTerritories = 'United\_Kingdom';\newline

\end{document}
